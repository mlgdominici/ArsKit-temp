%%
%%  This is file `secondadicop.tex', v0.1.2
%%  Stato: Bozza
%%
%%  Stato: Bozza
%%  Copyright 2006 Gruppo Utilizzatori Italiani di TeX
%%
%%  This work may be distributed and/or modified under the
%%  conditions of the LaTeX Project Public License, either
%%  version 1.3a of this license or (at your option) any
%%  later version.
%%  The latest version of the license is in
%%     http://www.latex-project.org/lppl.txt
%%
%%  Author: Massimo Caschili
%%          (massimo.caschili@tiscali.it)
%%
%%  This work has the LPPL maintenance status "author-maintained".
%%

\documentclass[]{arstexnica}
\usepackage[italian]{babel}
\usepackage[latin1]{inputenc}
\usepackage[T1]{fontenc}
\usepackage{lmodern}
\usepackage{microtype}
\usepackage{hyperref}
% \geometry{
%    hmargin=2.5cm,
%    vmargin={2cm,2.5cm},
%    bindingoffset=3mm,
%    columnsep=15pt
% }
% \usepackage{secondadicop}
%\usepackage{multicol}
% \usepackage{microtype}
% \makeatletter
% \documentclass[journal,final]{arstexnica}

% system packages
\usepackage[english,italian]{babel}
\usepackage[utf8x]{inputenc}
\usepackage[T1]{fontenc}
\usepackage{lmodern}
\usepackage{pdfpages}
\usepackage{pax}
\usepackage{hyperref}

% system definitions
\Urlmuskip = 0mu plus 1mu %% cfr `url.sty'

\makeatletter
\hypersetup{% Attenzione: hyperref è sempre caricato! Dal pacchetto `guit', chiamato da `arstestata', chiamato da `arstexnica'. Non è più vero almeno dal 2012 (v0.9.1 di guit.sty).
  pdfauthor   = {ArsTeXnica},
  pdftitle    = {ArsTeXnica, Numero \AT@number, \AT@year}
  pdfkeywords = {TeX, LaTeX, ConTeXt, XeLaTeX, tipografia digitale}
}
\makeatother

\begin{document}

\pdfstringdefDisableCommands{% per i bookmarks, nella versione elettronica.
  \let\thanks\@gobble
  \renewcommand{\thanksmark}[1]{}
  \renewcommand{\thanksgap}[1]{}
  \def\thetitle{\@title}
  \def\newline{\ }
  \def\\{\ }
}

\selectlanguage{italian}

% \setcounter{page}{-1}

%Struttura fissa-----------
\pagestyle{empty}
\IncludeSuppl[prima]{copertina/prima}
\IncludeSuppl[seconda]{copertina/seconda}

\pagestyle{journal}
\ATtableofcontents
%--------------------------

% Qui i vari articoli
% Per ora li metteremo a mano, presto in automatico

%\IncludeArticle{<directory>/<file_articolo>}

%--------------------------
% Qui l'eventuale riempitivo
% \includepdf{eventi.pdf}
%--------------------------

% Struttura fissa ---------
\ATbackmatter
\ATcleartoverso
\IncludeSuppl{copertina/colophon}
\IncludeSuppl{copertina/terza}
\IncludeSuppl[quarta]{copertina/quarta}
%--------------------------

\end{document}

% \makeatother
\begin{document}
\thispagestyle{empty}
%\begin{multicols}{2}
\noindent\logoseccop\par                %
\noindent\gguitext\par\medskip          %
\noindent\pres\par\medskip              %
\noindent\redazione                     %
%------------------\\%
%\finecolonna       %
%------------------\\%

\noindent \Ars{} � la prima rivista italiana dedicata a \TeX, a
\LaTeX{} ed alla tipografia digitale. Lo scopo che la rivista si
prefigge � quello di diventare uno dei principali canali italiani di
diffusione di informazioni e conoscenze sul programma ideato quasi
trent'anni fa da Donald Knuth.

Le uscite avranno, almeno inizialmente, cadenza semestrale e
verranno pubblicate nei mesi di Aprile e Ottobre. In particolare, la
seconda uscita dell'anno conterr� gli Atti del Convegno Annuale del
\guit, che si tiene in quel periodo.

La rivista � aperta al contributo di tutti coloro che vogliano
partecipare con un proprio articolo. Questo dovr� essere inviato
alla redazione di \Ars{}, per essere sottoposto alla valutazione di
recensori. � necessario che gli autori utilizzino la classe di
documento ufficiale della rivista; l'autore trover� raccomandazioni
e istruzioni pi� dettagliate all'interno del file di esempio
(\texttt{.tex}). Tutto il materiale � reperibile all'indirizzo web
della rivista.

Gli articoli potranno trattare di qualsiasi argomento inerente al
mondo di \TeX\ e \LaTeX\ e non dovranno necessariamente essere
indirizzati ad un pubblico esperto. In particolare tutorials,
rassegne e analisi comparate di pacchetti di uso comune, studi di
applicazioni reali, saranno bene accetti, cos� come articoli
riguardanti l'interazione con altre tecnologie correlate.

Di volta in volta verr� fissato, e reso pubblico sulla pagina web,
un termine di scadenza per la presentazione degli articoli da
pubblicare nel numero in preparazione della rivista. Tuttavia gli
articoli potranno essere inviati in qualsiasi momento e troveranno
collocazione, eventualmente, nei numeri seguenti.

Chiunque, poi, volesse collaborare con la rivista a qualsiasi titolo
(recensore, revisore di bozze, grafico, etc.) pu� contattare la
redazione all'indirizzo:

\begin{center}
\arsmail.
\end{center}

%\bigskip

\noindent\notacopy\par\bigskip
\noindent\contributi\par\bigskip        % Associati a GuIT
\noindent\textbf{\large Indirizzi}\par\medskip %
\noindent\guitind\par\medskip           % Indirizzo sede GuIT
\noindent\arsind\par\bigskip            % Indirizzo posta elettronica redazione
%\noindent\guitcopyright\par\bigskip     %
\noindent\issn\par\bigskip
\arsdatastampa{15}%
%\end{multicols}
\end{document}
